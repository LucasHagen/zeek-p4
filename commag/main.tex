% Project based on template.tex.

\documentclass[journal,comsoc]{IEEEtran} % For internal use
% \documentclass[journal,comsoc,pagebackref=false,bookmarks=false]{IEEEtran} % For submission

\input{ieee.tex}
\usepackage[utf8]{inputenc}     % pacote para acentuação

\usepackage{graphicx}           % pacote para importar figuras
\usepackage{svg}

\usepackage{times}              % pacote para usar fonte Adobe Times
\usepackage{verbatim}

\usepackage{csquotes}
\usepackage{hyperref}
\usepackage[alf]{abntex2cite}

\usepackage{subcaption} % for subfigures

% ----------------------------------------------------------------
% Code Listings
% 
% Provided by Alexandre Ilha (MSc Thesis)
% 
% Refs:
% https://en.wikibooks.org/wiki/LaTeX/Source_Code_Listings
% https://ctan.org/pkg/listings
% https://mirrors.ctan.org/macros/latex/contrib/listings/listings.pdf

\usepackage{listings}

\lstdefinelanguage{P4}[]{C++}{
    morekeywords={accept, action, actions, apply, bit, const, control, default, default_action, header, in, inout, key, out, packet_in, parser, select, size, start, state, struct, table, transition}
}

\lstdefinelanguage{Zeek}[]{C++}{
    morekeywords={addr, const, count, event, export, global, local, module, port, print, priority, redef, timeout, when},
    morecomment=[l]{\#}
}

\lstdefinestyle{code}{
    breaklines,
    basicstyle=\ttfamily\scriptsize,
    captionpos=t,
    floatplacement=htb,
    frame=tb,
    keywordstyle=\bfseries,
    numbers=left,
    numbersep=5pt,
    numberstyle=\ttfamily\tiny,
    showspaces=false,
    showstringspaces=false,
    tabsize=2
}

\lstdefinestyle{p4}{
    style=code,
    language=P4
}

\lstdefinestyle{zeek}{
    style=code,
    language=Zeek
}

\lstdefinestyle{inline}{
    breaklines,
    basicstyle=\ttfamily\normalsize
}

\lstset{style=code}

\newcommand{\code}[1]{\texttt{#1}}
% TODO figure out why this messes up line breaks.
% \renewcommand{\code}[1]{\lstinline[style=inline]{#1}}


% \NewCommandCopy{\oldtitle}{\title}
% \renewcommand{\title}[1]{%
%     \oldtitle{#1}%
%     \edef\thetitle{#1}%
% }

% \NewCommandCopy{\oldauthor}{\author}
% \renewcommand{\author}[2]{%
%     \oldauthor{#1}{#2}%
%     \edef\theauthor{#2~#1}%
% }

% \NewCommandCopy{\olddate}{\date}
% \renewcommand{\date}[2]{%
%     \olddate{#1}{#2}%
%     \edef\themonth{#1}%
%     \edef\theyear{#2}%
% }

% \NewCommandCopy{\oldlocation}{\location}
% \renewcommand{\location}[2]{%
%     \oldlocation{#1}{#2}%
%     \edef\thelocation{#1}%
% }

\begin{document}

% Titles are generally capitalized except for words such as a, an, and, as,
% at, but, by, for, in, nor, of, on, or, the, to and up, which are usually
% not capitalized unless they are the first or last word of the title.
% Linebreaks \\ can be used within to get better formatting as desired.
% Do not put math or special symbols in the title.
\title{Enabling Network Intrusion Detection\\at Data Plane Speeds}

% Author names and IEEE memberships.
% note positions of commas and nonbreaking spaces ( ~ ) LaTeX will not break
% a structure at a ~ so this keeps an author's name from being broken across
% two lines.
% use \thanks{} to gain access to the first footnote area
% a separate \thanks must be used for each paragraph as LaTeX2e's \thanks
% was not built to handle multiple paragraphs
\author{
  \IEEEauthorblockN{
    Lucas~Sonntag~Hagen,
    Alexandre~da~Silveira~Ilha,
    Jonatas~Adilson~Marques,\\
    and Luciano~Paschoal~Gaspary,~\IEEEmembership{Senior~Member,~IEEE}\\
  }
  \IEEEauthorblockA{
    Institute of Informatics,
    Federal University of Rio Grande do Sul - Brazil\\
    \{lucas.hagen, asilha, jamarques, paschoal\}@inf.ufrgs.br
  }
  \thanks{Manuscript submitted X NN, 2022.}
}

\markboth{IEEE Communications Magazine}{}

\maketitle

\input{content/00-abstract.tex}

\begin{IEEEkeywords}
software-defined networks, security, prototype implementation, testbed experimentation.
\end{IEEEkeywords}

\section{Introduction} 
\label{sec:introduction}
% 1 column.

\IEEEPARstart{T}{his} paper rocks! P4~\cite{Bosshart2014} rocks!

\blindtext[2]


\section{Background -- Identifying Candidate Operations for Offloading}
\label{sec:background}

% 1 column.
% * Brief text including layers.
% * Table.

\begin{itemize}
    \item Brief text including layers.
    \item Table.
\end{itemize}

\blindtext[3]

\newpage
\section{WXYZ} 
\label{sec:architecture}

% 4 columns
% * Introductory text to general architecture.
% * Figure.
% * Brief description of architecture components and interactions.

\begin{itemize}
    \item Introductory text to the general architecture.
    \item Figure and explanation.
    \item Brief description of architectural components and interactions.
\end{itemize}

\blindtext[8]

\newpage
\section{Automatic Code Generation for WXYZ} 
\label{sec:code-generation}

% 4 columns.
% * Introductory text to automatic code generation.
% * Figure (workflow) and corresponding explanation.
% * Description of the code generation process.
% * A working example

\begin{itemize}
    \item Introductory text to automatic code generation.
    \item Figure about the workflow and the corresponding explanation.
    \item Description of the code generation process.
    \item A working example. 
\end{itemize}

\blindtext[8]

\newpage
\section{Evaluation} 
\label{sec:evaluation}

% 2,5 columns
% * Experimental setup (prototypes, testbed)
% * Functional evaluation
% * Performance evaluation

% What do we want to demonstrate? 
% Fn. Eval. - The solution can detect events of interest.
% Perf. Eval. - The solution is scalable. 


\blindtext[8]

\newpage
\chapter{Conclusion}
\label{cap:conclusion}

%contexto, proposta

%principais resultados

%future work

Conclusion...

\section*{acknowledgments}
This work was partially funded by the National Council for Scientific and Technological Development (CNPq - 441892/2016-7), the Coordination for the Improvement of Higher Education Personnel (CAPES - Finance Code 1), the São Paulo Research Foundation (FAPESP - 15/24494-8), the Brazilian National Research and Educational Network (RNP), and the National Science Foundation (NSF - CNS-1740911).

\bibliographystyle{IEEEtran}
\bibliography{IEEEabrv,references}

\vfill
\newpage      

\begin{IEEEbiographynophoto}{Lucas Sonntag Hagen}
Biography text here.
\end{IEEEbiographynophoto}

\begin{IEEEbiographynophoto}{Alexandre da Silveira Ilha}
Biography text here.
\end{IEEEbiographynophoto}

\begin{IEEEbiographynophoto}{Jonatas Adilson Marques}
Biography text here.
\end{IEEEbiographynophoto}

\begin{IEEEbiographynophoto}{Luciano Paschoal Gaspary}
Biography text here.
\end{IEEEbiographynophoto}

\end{document}
