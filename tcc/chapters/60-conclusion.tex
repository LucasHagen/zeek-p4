\chapter{Conclusion}
\label{cap:conclusion}

In this project, we investigated the benefits of using Programmable Data Planes to offload Zeek monitoring scripts. We also took the first step towards an automatic code generation mechanism, which enables any network operator without programming knowledge to offload Zeek scripts to programmable forwarding devices. We implemented an automatic code generator that identifies which Zeek Events are required by a set of scripts and, using templates, automatically generates P4 and Zeek code to offload these scripts.

After proposing additions to the RNA framework and implementing our prototypical automatic code generator, we evaluated the proposed approach and assessed its capabilities of automatically generating code and enhancing performance. We showed the mechanism generates almost $3$ thousand lines of code, which, otherwise, a developer would need to write manually in order to offload four Zeek Scripts. We demonstrated that RNA can give a performance benefit compared to server-based intrusion detection, resulting in $57\times$ less CPU usage and $4\times$ less memory usage for the workload used in the experiments. Moreover, we have also shown that our approach can produce these benefits for network operators without any previous P4 programming knowledge. It is also important to note that these results are still to be confirmed with future experiments using hardware PFDs.

%\vspace{-0.5em}

%\section{Challenges and Difficulties}

%\vspace{-0.5em}
% "Reflexão pessoal", desafios do projeto (1 parágrafo)

%This project's main challenge was adapting an existing IDS, in our case Zeek, to work with PDP offloading. This adaptation was also challenging due to the lack of internal documentation on the Zeek internal subsystems, initially designed to facilitate adding support to new protocols without needing to modify the existing data flow.

% \section{Future Work}

This project took the first step towards adding automatic code generation to RNA. Next, we describe some opportunities for future work and items that could be improved. These items are both suggestions for RNA and the code generator mechanism.

\subsubsection*{Stateful Connections}

The next step to allow better protocol handling, mainly of TCP connections, is the implementation of stateful analysis into RNA. Stateful analysis will allow far more significant performance benefits if executed in the data plane. It will offload a significant part of Zeek state management, allowing P4 to analyze more connection-based protocols.


\subsubsection*{Multiple Offloaders}

At the current state of RNA and our code generation mechanism, it is not possible to trigger more than one Offloader per incoming packet. This limitation could impact future deployments where many scripts are being executed. This is a problem that should be solved in the future.


\subsubsection*{Enhanced Zeek \textit{Connection} Management}

Zeek uses an object called \textit{Connection} (not to be confused with an actual \textit{network connection}) to manage sessions internally. This object is not deallocated after usage in our aproach. Future work must develop a mechanism to handle protocol timeouts and free resources after these are no longer needed. Some protocols also have timeout events, which should also be triggered by the PDP.

