\chapter{Proof of Concept and Evaluation}
\label{cap:evaluation}

% Introduction...



\section{Implementation}
\label{sec:evaluation:implementation}
% Probably a small section, be careful not to be redundant (max 2 pages)

% ----------------------------------------
% moved from code generation chapter


% Describe how it is actually used to generate the code and then to deploy

% After verifying all \Offloaders{} and \ProtocolTemplates{} required to support the offloading of the desired scripts were provided, the code generation and merge procedure begins. Here is an example of usage:

% \begin{figure}[htb]
%     \centering
%     \caption{RNA Code Generation}
%     \lstinputlisting[style=Bash]{code/cmd_example.txt}
%     \label{code:cmd_example}
% \end{figure}

% \begin{figure}[htb]
%     \centering
%     \caption{RNA Tool - Usage}
%     \lstinputlisting[style=Bash]{code/cmd_usage.txt}
%     \label{code:cmd_usage}
%     %\vspace{-1em}
% \end{figure}

\section{The Prototype in Action} % MAYBE!!! (max 1-2 pages)


\section{Evaluation}
\label{sec:evaluation:evaluation}

% Metrics:
% 1. How many scripts we can run (without intervention)
% 2. How many lines were produced per script
%     a. Include copied from the template (maybe create a table with the outline)
% 3. Performance gain in offloading (optional): compare a raw data set vs mRNA messages data set.
%     a. memory, cpu, execution time



