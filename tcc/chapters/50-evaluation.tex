\chapter{Proof of Concept and Evaluation}
\label{cap:evaluation}

% Introduction...

In this chapter we present how the automatic code generation mechanism, proposed in \autoref{cap:code_gen}, was implemented in a fully functional proof of concept. We also describe how we evaluated the gain in performance, and ease of development and deployment.

To facilitate the development of this project and to keep the scope in a defined limit, we did not use P4 compatible switches, instead, we use software emulation to run the P4 code. This enabled us to develop with great efficiency the generation tool, and enabled us to test each interaction of the development, making sure the prototype was funcitonal.


\section{Implementation}
\label{sec:evaluation:implementation}
% Probably a small section, be careful not to be redundant (max 2 pages)

In this section, we describe in more details how whole development process of code generation tool. We divide this explanation into three different subsections. We first explain the structure of the Protocol and Offloader Templates, which are one of the main inputs for our tool. Then we explain the implementation aspects of our prototype. And to finalize, we explain how the output of our tool, the automatically generated code, is deployed in a virtualized network, using an emulation of a P4-switch.


\subsection{Protocol and Offloader Templates}


\subsection{Code generation}


\subsection{Deployment}


% 1. Explain how we implemented RNA and the code generation procedure (focando em aspectos mais arquiteturais e tecnologias empregadas. Os *why*s fazem parte do texto até para justificar escolhas mais importantes, mas não são o foco.)

% 2. Explain why we used Python (not that important)

% 3. Explain the master template and the markers (how they are replaced)

% 4. How the code generation is done (using a tree)

% 5. Explain how it is deployed with p4app







% ----------------------------------------
% moved from code generation chapter

% After verifying all \Offloaders{} and \ProtocolTemplates{} required to support the offloading of the desired scripts were provided, the code generation and merge procedure begins. Here is an example of usage:

% \begin{figure}[htb]
%     \centering
%     \caption{RNA Code Generation}
%     \lstinputlisting[style=Bash]{code/cmd_example.txt}
%     \label{code:cmd_example}
% \end{figure}

% \begin{figure}[htb]
%     \centering
%     \caption{RNA Tool - Usage}
%     \lstinputlisting[style=Bash]{code/cmd_usage.txt}
%     \label{code:cmd_usage}
%     %\vspace{-1em}
% \end{figure}

\section{The Prototype in Action} % MAYBE!!! (max 1-2 pages)


\section{Evaluation}
\label{sec:evaluation:evaluation}

% Metrics:
% 1. How many scripts we can run (without intervention)
% 2. How many lines were produced per script
%     a. Include copied from the template (maybe create a table with the outline)
% 3. Performance gain in offloading (optional): compare a raw data set vs mRNA messages data set.
%     a. memory, cpu, execution time


% - FTP Bruteforcing (done)
% - Pingback (done): https://github.com/corelight/pingback
% - Detect traceroute (theoretically done)
% - NTP Monlist

