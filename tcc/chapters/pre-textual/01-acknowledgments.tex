\chapter*{Acknowledgements}

Acknowledgements...

\chapter*{Agradecimentos}

Agradeço aos meus pais Rejane Sonntag Hagen e Roberto Hagen por me proporcionarem uma educação de qualidade e me apoiarem em todos os momentos. Ao meu irmão Bruno e minha namorada Giulia que fizeram parte dessa trajetória, incentivando-me a seguir em frente e a não desistir.

Ao meu orientador, Prof.~Dr.~Luciano Paschoal Gaspary, que me guiou nessa árdua tarefa chamada TCC, sempre com paciência e sabedoria, compartilhando um pouco de todo o seu conhecimento e sua experiência. Ao doutorando MSc.~Jonatas Adilson Marques e ao MSc.~Alexandre da Silveira Ilha por compartilharem seu tempo sanando as minhas dúvidas e guiarem-me em minha pesquisa.

Agradeço também à intituição UFRGS e seus professores, onde iniciei uma jornada profissional, conheci grandes amigos e construí meu currículo. Sou imensamente grato por todas as oportunidades recebidas. Faço aqui menção dos professores Raul e Taisy Weber, Lucas Schnorr e Gabriel Nazar os quais tiverem um participação especional nessa jornada acadêmica.

Além disso, gostaria de agradecer aos meus amigos e colegas de curso, os quais foram muito importantes em minha trajetória acadêmica. Meciono, em especial, meus colegas de intercâmbio Bernardo, Emily e Iron, os quais se tornaram uma família no ano de 2020, durante o auge da Pandemia da COVID-19. Ainda no intercâmbio, agredeço ao meu orientador Prof.~Dr.~Reinhard Gotzhein e ao doutorando MSc.~Paulo Aragão, que me acolheram e mentoraram a minha passagem pela Technische Universität Kaiserslautern. 

Ademais, gostaria de mencionar os meus chefes de estágios e outros colegas de trabalho, com os quais dividi momentos de apredizado e de lazer: Lauro Souza e Breno Araújo, do estágio que realizei no Google Brasil no ano de 2021-2022; Daniel Thiel, Hélio Fuques e Hernandi Krames do estágio na AEL Sistemas no ano de 2018-2020.

Por fim, agradeço aos demais professores, mentores, familiares e amigos, que direta e indiretamente participaram da minha jornada acadêmica, transmitindo aprendizados, compatilhando ideias e bons momentos, dividindo momentos de angústia e também de alegria.

\clearpage