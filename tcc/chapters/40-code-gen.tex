\chapter{Automatic Code Generation for RNA}
\label{cap:code_gen}

% Introduction

In this chapter, we present a tool for automatic code generation for the RNA Framework. This tool enables network operators to deploy the Framework without having the experience and knowledge to develop software for Zeek or programmable forwarding devices (P4). To support this tool, in the previous chapter we proposed additions to the RNA Framework.

This tool started with the idea of having a set of Zeek scripts as input, and two components on the output: a Zeek Script/Package and a P4 program. This became evidently unfeasible for the scope of this project due to the complexity and the logic behind each event triggering logic in Zeek. This obstacle steered us into another approach. The proposed tool uses both previously mentioned concepts, the \ProtocolTemplates{} and the \Offloaders{}, as a source of templates and resources, in order to implement the software required to offload the desired scripts. Another proposed change to the design is the generation of a unique Zeek Plugin, which will automatically deploy the P4 code when initiated, instead of having two separate deployable components.

It is important to mention when we talk about offloading scripts, we are not offloading the processing of the scripts that happen on the PSI (see Section \ref{sec:bg:zeek_psi}). We offload the steps and procedures required to transform network packets into the structures that are needed to run these scripts, usually Zeek Events. Those operations selected for offloading, which are described in Section \ref{sec:bg:zeek_candidate_operations} and also analyzed in-depth by Ilha \cite{Ilha2022}, make the system more performative and able to support more network throughput in this world of constantly increasing network speeds.

\section{Overview}
\label{sec:code_gen:overview}

This section presents an overview of the RNA Code Generation Tool, starting with its inputs and expected output, which is illustrated in Figure \ref{fig:code_gen_black_box}. The main input for this tool is the set of scripts to be offloaded. These scripts need to be given in full, including their source code so the Tool can extract what events they monitor.

\begin{figure}[htb]
    \caption{RNA - Code Generator Tool - Inputs and Outputs}
    \begin{center}
        \includegraphics[width=0.98\textwidth]{images/code_gen_black_box.pdf}  
    \end{center}
    \label{fig:code_gen_black_box}
    \legend{Source: the author}
\end{figure}

The second set of inputs are what we call templates, both \ProtocolTemplates{} and \Offloader{} Templates. They are a pool of known implementations of protocols and events that can be used for offloading scripts. A template being present in this pool doesn't mean it will be included in the final output, but it means it is available in case a script needs its implementation.

The desired output of our Tool is a single Zeek Plugin following the structure previously presented in Section \ref{sec:rna:overview}, illustrated in Figure \ref{fig:arch_low_level}. This Zeek Plugin when executed will: configure the switch, create a mirroring session for the Zeek monitoring system, deploy the P4 code, and register all Translators in Zeek's Event Engine. This eliminates the need for the operator to coordinate the deployment of two separate systems, the RNA Host Engine, and the RNA Switch Engine.

To be able to execute this task, the first objective of the RNA Tool is to parse all the provided Zeek scripts and identify which are the observed events in every script. Once this pool of events is created, the Tool starts searching for an Offloader (from the templates pool) that is capable of offloading this event. This step is finished when every event has a corresponding \Offloader{}. It is also important to note that one Offloader may offload more than one event at a time, which is not a problem.

After all \Offloaders{} have been selected, the Tool searches for all protocols templates required by these \Offloaders{}. To ensure all protocols used by the \Offloaders{} can be parsed, a graph structure is generated using all available template protocols. The Tool then ensures there is a path from the Ethernet protocol, which is our root protocol, to every protocol where an \Offloader{} is registered. If this process succeeds, the program executes some optimizations and heuristics (described in a later section) and generates the final code.

Some of the resulting code that is present in the final Zeek Plugin is merged from the templates, and not completely generated by our tool. It is not yet possible to fully generate all offloader code, because the process of converting C++ or P4 code from one to another is very complicated and is out of the scope of this project. As explained in the previous section, while our code generator is not able to fully generate all required code, some of it must be included in the templates.

\section{Implementation Details}
\label{sec:code_gen_impl}

% Describe how to actually works and maybe some implementation details